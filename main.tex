\documentclass[portrait, noFonts]{betterposter/betterposter}
\usepackage[type=poster, math=fancy, font=palatino, osf]{styles/style}

% Even smaller actions.
\makeatletter
\renewcommand*{\@smallTriangle}[2]{\raisebox{\depth+0.070em}{\scalebox{0.65}{\ensuremath{#1#2}}}}
\renewcommand*{\ract}{\mathbin{\mathpalette\@smallTriangle\lhd}}
\makeatother

% Tweak betterposter's default font sizes.
\renewcommand{\fontsizestandard}{\fontsize{39.00}{55.50} \selectfont}  % Bit bigger
\renewcommand{\fontsizeauthor}{\fontsize{43.00}{63.00} \selectfont}    % Bit smaller
\renewcommand{\fontsizetitle}{\fontsize{51.00}{100.00} \selectfont}    % Bit smaller
\renewcommand{\fontsizemain}{\fontsize{135.00}{200.00} \selectfont}    % Much bigger
\newcommand{\fontsizeinstitution}{\fontsize{30.00}{50.00} \selectfont} % Bit smaller

% A colourful box to (hopefully) grab the reader's attention.
\usepackage[most]{tcolorbox}
\colorlet{highlightColour}{red!55!black}
\NewTColorBox{highlight}{O{}}{%
  empty,
  title={#1},
  attach boxed title to top left,
  boxed title style={empty,size=minimal,toprule=0pt,top=4pt,left=3.5mm,bottom=5pt,overlay={}},
  coltitle=highlightColour,
  fonttitle=\bfseries,
  before=\par\medskip\noindent,
  bottom=0pt,
  overlay unbroken={%
    \draw[highlightColour,line width=3pt]
      % Attach to title if any, otherwise attach to frame.
      ([xshift=-0pt] \IfBlankTF{#1}{frame}{title}.north west)
      --
      ([xshift=-0pt] frame.south west);%
  },
}

% This is a little bit more visible on a poster.
\renewcommand{\k}{\mathbb{k}}

% These haven't made it into the style file yet.
\NewDocumentCommand{\tetramod}{O{} O{} O{} O{}}{%
  \leftidx{^{#1}_{#2}}{(\kVect)}{_{#3}^{#4}}%
}
\NewDocumentCommand{\tetramodfd}{O{} O{} O{} O{}}{%
  \leftidx{^{#1}_{#2}}{(\kvect)}{_{#3}^{#4}}%
}
\newcommand*{\trimod}[1][B]{\tetramodfd[#1][#1][][#1]}
\newcommand*{\bcomod}[1][B]{\tetramodfd[#1]}
\renewcommand*{\hom}[1]{\ensuremath{\lfloor#1\rfloor}}
\newcommand*{\cohom}[1]{\ensuremath{\lceil#1\rceil}}

\begin{document}

%%% TOP
\betterThreeColumns{                                           % LEFT
  \title{Lax Module Functors, Reconstruction, and Hopf Algebras}
  \vspace{-1.0em}
  \author{Tony Zorman {\hfill\large Based on joint work with}}
  \vspace{0.3em}
  \institution{\fontsizeinstitution{}TU Dresden {\hfill\large Mateusz Stroiński}}

  \bigskip\bigskip
  Our investigation starts with Kelly's classic notion of \emph{doctrinal adjunctions}:\\[-0.6em]
  \begin{highlight}
    For monoidal categories \(\cat{C}\) and \(\cat{D}\), and an adjunction \(\stdadj\),
    there is a bijection between oplax monoidal structures on \(F\) and lax monoidal structures on \(U\).
  \end{highlight}

  This result provides a kind of \emph{Tannaka reconstruction} for bimonads,
  as given by Moerdijk.\\[-0.6em]
  \begin{highlight}
    There is a bijective correspondence between opmonoidal structures of a monad \(T\) on a monoidal category \(\cat{C}\),
    and mon\-oidal structures on \(\cat{C}^T\) such that the forgetful functor is strong monoidal.
  \end{highlight}
}{                                                             % MIDDLE
  \bigskip
  This correspondence lifts to the setting of \emph{oplax \(\cat{C}\)-module monads}: monads \(T\) on \(\cat{M}\), for a (left) \(\cat{C}\)-module category \(\cat{M}\),
  that are equipped with a natural \emph{action} morphism, resembling that of an opmonoidal comultiplication:
  \[
    T_{\mathsf{a}} \from T(\blank \lact \bblank) \,\nt\, \blank \lact T(\bblank).
  \]

  \vspace{0.5em}\begin{highlight}[Theorem (Halbig--Z)]
    There is a bijective correspondence between
    oplax \(\cat{C}\)-module structures of a monad \(T\) on \(\cat{M}\),
    and \(\cat{C}\)-module structures on \(\cat{M}^T\) such that the forgetful functor \(U^T\) is a strict \(\cat{C}\)-module functor.
  \end{highlight}

  In contrast to these results stands \emph{Deligne reconstruction},
  where one does not require a forgetful functor—%
  at the cost of only recovering the algebraic object of interest up to \emph{Morita equivalence}.
}{                                                             % RIGHT
  \bigskip
  Furthermore, the monads we consider are naturally \emph{lax} module functors.
  In that case, one obtains a \(\cat{C}\)-module structure on the \emph{Kleisli category} \(\cat{M}_T\) of \(T\).
  Under mild additional assumptions,
  this induces a unique \(\cat{C}\)-module structure on \(\cat{M}^T\).\\[-0.6em]
  \begin{highlight}[Proposition (Stroiński--Z)]
    Given a left \(\cat{C}\)-module structure on \(\cat{M}_T\),
    there is, up to isomorphism at most one left \(\cat{C}\)-module structure on \(\cat{M}_T\),
    such that the canonical inclusion \(\iota\from \cat{M}_T \to \cat{M}^T\) is a strong \(\cat{C}\)-module functor.
  \end{highlight}
  Let us now concentrate on the representation theoretic case.
  For a field \(\k\), suppose that \(\cat{C}\) is a \(\k\)-linear abelian monoidal category,
  and that \(\cat{M}\) is a \(\k\)-linear abelian left \(\cat{C}\)-module category.
}

%%% MIDDLE
\maincolumn{
  \bigskip\bigskip\bigskip\bigskip
  \begin{minipage}{\textwidth}
    \hspace*{0.1em}\begin{minipage}{0.75\textwidth}
      \centering
      Deligne reconstruction works\\[-0.6em]
      for nice module categories\\[-0.6em]
      over nice abelian bases\\[-0.6em]
    \end{minipage}\begin{minipage}{\textwidth}
      {\hspace{0.2em}{\LARGE Paper, Poster, References}\\[0.1em]
        \mbox{\hspace{0.45em}\qrcode[nolinks, height=13cm]{https://tony-zorman.com/ferrara2024}}}
    \end{minipage}
  \end{minipage}
  \bigskip\bigskip\bigskip\bigskip
}

%%% BOTTOM
\betterThreeColumns{                                           % LEFT
  An important ingredient in our study of the module structure of \(\cat{M}^T\)
  are \emph{internal} projective objects: objects
  \(X \in \cat{M}\) such that acting with any projective in \(\cat{C}\) is a projective in \(\cat{M}\).
  If \(X\) is \emph{closed}—the adjunction
  \[
    \adj{\blank \lact X}{\hom{X, \blank}}{\cat{C}}{\cat{M}}
  \]
  exists---this guarantees the right adjoint to be an exact functor.

  If \(X\) satisfies the additional condition of being a \emph{\(\cat{C}\)-generator},
  then \(\hom{X, \blank}\) even reflects zero objects;
  in particular, all of the preconditions of \emph{Beck's monadicity theorem}
  for abelian categories hold:

  \vspace{0.5em}\begin{highlight}
    An adjunction between abelian categories is monadic
    if and only if
    the right adjoint is right exact and reflects zero objects.
  \end{highlight}

  Naturally, one could instead talk about internal \emph{injective} objects
  and \(\cat{C}\)-cogenerators.
  This involves studying the adjunction
  \[
    \adj{\cohom{X, \blank}}{\blank \lact X}{\cat{M}}{\cat{C}}.
  \]

  \vspace{0.5cm} Putting all of these pieces together,
  we obtain a Deligne-type reconstruction result.
}{                                                             % MIDDLE
  \begin{highlight}[Theorem (Stroiński--Z)]
    If \(\cat{C}\) has enough projectives,
    then all \(\cat{C}\)-module categories with enough projectives that
    have a closed \(\cat{C}\)-projective \(\cat{C}\)-generator
    are of the form \(\cat{C}^{\hom{X,\, \blank \,\lact\, X}}\).
  \end{highlight}

  This theorem in particular does not need a rigidity assumption.
  If this is added, the statement reduces from the monadic to the algebraic case.

  \vspace{0.5em}\begin{highlight}
    If \(\cat{C}\) has enough projectives and is rigid,
    then for all \(\cat{C}\)-module categories \(\cat{M}\) with enough projectives that
    have a closed \(\cat{C}\)-projective \(\cat{C}\)-generator,
    there exists an algebra object \(A \in \cat{C}\)
    with \(\mathrm{mod}_{\cat{C}}(A) \simeq \cat{M}\).
  \end{highlight}

  A category having enough injectives may instead be replaced by considering its \emph{ind-completion}.
  Hopf algebraically,
  this yields a variant of a result by Ostrik.

  \vspace{0.5em}\begin{highlight}[Corollary (Stroiński--Z)]
    Every finite abelian \(\tetramodfd[H]{}\)-module category \(\cat{M}\),
    with \(\blank \lact M\) exact for all \(M \in \cat{M}\),
    is equivalent to \(\mathsf{comod}_{\tetramodfd[H]{}}(C)\), for an \(H\)-comodule algebra \(C\).
  \end{highlight}
}{                                                             % RIGHT
  We also obtain a version of the fundamental theorem of Hopf modules
  for the case of \emph{Hopf trimodules}.
  The statement is akin to the quasi-bialgebraic case, as proven by Hausser--Nill and Saracco.

  \vspace{0.8em}\begin{highlight}[Proposition (Stroiński--Z)]
    A bialgebra \(B\) admits a twisted antipode
    if and only if
    the natural arrow
    \[
      B \otimes \blank \from \tetramod[B] \to \tetramod[B][B][][B]
    \]
    is an equivalence.
  \end{highlight}

  Lastly, the \emph{fusion operators} of a bimonad
  in the sense of Bruguières--Lack--Virelizier
  also fit into this framework%
  ---they can be seen as coherence morphisms for a natural module action.

  \vspace{0.8em}\begin{highlight}[Proposition (Stroiński--Z)]
    Let \(\adj{F}{U}{\cat{C}}{\cat{D}}\) be an opmonoidal adjunction.
    The bimonad \(T \eqdef UF\) on \(\cat{C}\) is Hopf
    if and only if
    the coherence cells for the natural oplax \(\cat{C}^T\)-module monad structure on \(T\) are isomorphisms.
  \end{highlight}
}

\end{document}

%%% Local Variables:
%%% mode: LaTeX
%%% TeX-master: t
%%% End:
